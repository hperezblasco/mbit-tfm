% Chapter 1

\chapter{Objectives} % Main chapter title

\label{Chapter1} % For referencing the chapter elsewhere, use \ref{Chapter1} 

%----------------------------------------------------------------------------------------

% Define some commands to keep the formatting separated from the content 
\newcommand{\keyword}[1]{\textbf{#1}}
\newcommand{\tabhead}[1]{\textbf{#1}}
\newcommand{\code}[1]{\texttt{#1}}
\newcommand{\file}[1]{\texttt{\bfseries#1}}
\newcommand{\option}[1]{\texttt{\itshape#1}}

%----------------------------------------------------------------------------------------

\section{Problem Description}
Aircraft lifetime and maintenance is a recurrent problem for companies due to its high costs.
Being able to predict when an aircraft is going to have failures or malfunctioning is a key topic for improving costs and performance of this machines.

The scenario uses data from simulated aircraft sensor to predict when an aircraft engine will fail in the future.
Two different approaches are used for this problem:

\begin{itemize}
\item A regression models that manages to answer the question: How many more cycles an in-service engine will last before it fails?
\item A binary classification model that manages to answer the question: Is this engine going to fail within a specific number of time cycles?
\end{itemize}

%----------------------------------------------------------------------------------------

\section{Data Summary}\label{sec:data-summary}

The data provided is divided in three different sets:

\subsection{Training data}\label{subsec:training-data}

The training data consists of multiple multivariate time series with "cycle" as the time unit, together with 21 sensor readings for each cycle.
Each time series can be assumed as being generated from a different engine of the same type.
Each engine is assumed to start with different degrees of initial wear and manufacturing variation, and this information is unknown to the user.
In this simulated data, the engine is assumed to be operating normally at the start of each time series.
It starts to degrade at some point during the series of the operating cycles.
The degradation progresses and grows in magnitude.
When a predefined threshold is reached, then the engine is considered unsafe for further operation.
In other words, the last cycle in each time series can be considered as the failure point of the corresponding engine.
Taking the sample training data shown in the following table as an example, the engine with id=1 fails at cycle 192.

\begin{table}
\caption{Training Data}
\label{tab:training-data}
\centering
\begin{tabular}{l l l l l l l l l }
\toprule
\tabhead{Id} & \tabhead{Cycle} & \tabhead{Setting 1} & \tabhead{Setting 2} & \tabhead{S1} & \tabhead{S2} & \tabhead{S3} & \tabhead{...} \\
\midrule
1 & 1   & -0.0007  & -0.0004 & 100.0 & 518.67 & 641.82 & ... \\
1 & 2   & 0.0019   & -0.0003 & 100.0 & 518.67 & 642.15 & ... \\
1 & 3   & -0.0043  & 0.0003  & 100.0 & 518.67 & 642.35 & ... \\
... & ... & & & &  & & ... \\
1 & 191 & 0        & -0.0004 & 100.0 & 518.67 & 643.34 & ... \\
1 & 192 & 0.0009   & 0       & 100.0 & 518.67 & 643.54 & ... \\
2 & 1   & -0.0018  & 0.0006  & 100.0 & 518.67 & 641.89 & ... \\
2 & 2   & 0.0009   & -0.0003 & 100.0 & 518.67 & 641.82 & ... \\
2 & 3   & 0.0018   & 0.0003  & 100.0 & 518.67 & 641.55 & ... \\
... & ... & & & &  & & ... \\
\bottomrule\\
\end{tabular}
\end{table}

\subsection{Text data}\label{subsec:text-data}

The testing data has the same data schema as the training data.
The only difference is that the data does not indicate when the failure occurs (in other words, the last time period does NOT represent the failure point).
Taking the sample testing data shown in the following table as an example, the engine with id=1 runs from cycle 1 through cycle 31.
It is not shown how many more cycles this engine can last before it fails.

\begin{table}
\caption{Test Data}
\label{tab:test-data}
\centering
\begin{tabular}{l l l l l l l l l }
\toprule
\tabhead{Id} & \tabhead{Cycle} & \tabhead{Setting 1} & \tabhead{Setting 2} & \tabhead{S1} & \tabhead{S2} & \tabhead{S3} & \tabhead{...} \\
\midrule
1 & 1  & 0.0023   & 0.0003  & 100.0 & 518.67 & 643.02 & ... \\
1 & 2  & -0.0027  & -0.0003 & 100.0 & 518.67 & 641.71 & ... \\
1 & 3  & 0.0003   & 0.0001  & 100.0 & 518.67 & 642.46 & ... \\
... & ... & & & &  & & ... \\
1 & 30 & -0.0025  & 0.0004  & 100.0 & 518.67 & 642.79 & ... \\
1 & 31 & -0.0006  & 0.0004  & 100.0 & 518.67 & 642.58 & ... \\
2 & 1  & -0.0009  & 0.0006  & 100.0 & 518.67 & 641.89 & ... \\
2 & 2  & -0.0011  & 0.0002  & 100.0 & 518.67 & 642.51 & ... \\
2 & 3  & 0.0002   & 0.0003  & 100.0 & 518.67 & 642.58 & ... \\
... & ... & & & &  & & ... \\
\bottomrule\\
\end{tabular}
\end{table}

\subsection{Ground truth data}\label{subsec:ground-truth-data}
The ground truth data provides the number of remaining working cycles for the engines in the testing data.
Taking the sample ground truth data shown in the following table as an example, the engine with id=1 in the testing data can run another 112 cycles before it fails.

\begin{table}
\caption{Ground Truth Data}
\label{tab:ground-truth-data}
\centering
\begin{tabular}{l}
\toprule
\tabhead{RUL} \\
\midrule
112 \\
98 \\
69 \\
82 \\
91 \\
... \\
\bottomrule\\
\end{tabular}
\end{table}

%----------------------------------------------------------------------------------------

\section{Dissertation goals}

Once the data and the problem to solve is known, the goal of this dissertation is to expose and analyze the different approaches and alternatives used for the problem resolution.

All of the different methodologies, technical solutions and results will be explained step by step and compared within each other.

Finally some conclusions must be extracted in order to clarify which one is the best approach to solve the problem concerning this dissertation context.

